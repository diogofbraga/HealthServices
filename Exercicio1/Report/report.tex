\documentclass[a4paper]{article}

\usepackage[utf8]{inputenc}
\usepackage[portuges]{babel}
\usepackage{indentfirst}
\usepackage{graphicx}
\usepackage{float}
\usepackage{caption}
\usepackage{subcaption}
\usepackage[T1]{fontenc}
\usepackage{listings}
\usepackage{amsmath}
\usepackage{mathtools}
\renewcommand{\familydefault}{\sfdefault}

\title{Sistemas de Representação de Conhecimento e Raciocínio - Programação em lógica e Invariantes}
\author{Diogo Braga A82547 \and João Silva A82005 \and Ricardo Caçador A81064 
\and Ricardo Milhazes A81919}
\date{\today}

\begin{document}

\maketitle

\begin{abstract}
O trabalho representado neste relatório foi desenvolvido no âmbito da UC de Sistemas de Representação de Conhecimento e Raciocínio por forma a desenvolver competências na utilização da linguagem de programação em lógica - PROLOG.
Este exercício consistiu no desenvolvimento de uma base de conhecimento e racíocinio para caracterizar um universo de discurso na área da prestação de cuidados de saúde.
Este relatório irá explicar todo o processo que envolvou a a criação dessa base até ao resultado final.
\end{abstract}

\tableofcontents

\newpage


\section{Introdução}
\label{sec:intro}

\section{Conclusão}
\label{sec:conclusao}




\end{document}
